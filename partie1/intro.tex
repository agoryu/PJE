\section*{Introduction}
\addcontentsline{toc}{section}{Introduction}

Dans le cadre de notre formation en licence informatique à l'université de Lille 1, nous nous sommes orienté vers le projet permettant
de découvrir l'une des spécialités de master. Nous avons sélectionné le projet ``Interface Multi-Touch'' qui correspond à
la spécialité IVI\footnote{Image Vision Interaction}. Ce projet a pour but de réaliser une interface tactile pour interagir avec un système permettant de
manipuler des images sur un ordinateur. Ce rapport reprend l'ensemble des connaissances acquises durant la première partie 
du projet qui correspond à des travaux pratiques manipulant la caméra utilisée pour l'interface multi-touch. Durant ce rapport, 
nous allons expliquer : \\
\begin{itemize}
 \item comment fonctionne la formation d'une image,
 \item comment la caméra peut acquérir de l'information et la numériser,
 \item comment récupérer des informations sur les doigts de l'utilisateur,
 \item et omment différencier les différents doigts de l'utilisateur.\\
\end{itemize}

Pour répondre à ces questions, nous allons d'abord détailler les éléments d'optiques mis en oeuvre dans ce projet. Puis, nous allons
utiliser les bibliothèques d'Opencv pour découvrir comment récupérer et manipuler des images provenant de la caméra. Ensuite, nous verrons
comment appliquer des filtres à des images. Et enfin, nous détaillerons le programme permettant de récupérer des informations sur 
les doigts de l'utilisateur.
